\section{Modules}

\noindent\textbf{Introduction:} corbat has one main module, \texttt{recorder\_full} typically imported as \texttt{rec}. Its dependencies are:
\begin{enumerate}[noitemsep]
	\item asyncio, time for asynchronous methods,
	\item datetime, to work with datetime objects,
	\item copra.websocket, to connect to coinbase,
	\item pandas, to store out time series arrays, 
	\item LOB\_funcs.py, to manage messages and create time series arrays, 
	\item history\_funcs.py, for some helper functions, and 
	\item gc and numpy. 
\end{enumerate}
The modules, and their dependencies are listed in the tree diagram below.\\

\dirtree{%
	.1 CSV\_out\_test.py.
	.2 recorder\_full.py.
	.3 asyncio, time.
	.3 dateime.
	.3 copra.websocket.
	.3 pandas(note can move this later).
	.3 LOB\_funcs.py.
	.4 pandas.
	.4 copy.
	.4 bisect.
	.4 numpy.
	.4 history\_funcs.py.
	.3 history\_funcs.py.
	.4 pandas.
	.4 bisect.
	.3 gc.
	.3 numpy.
	.2 datetime.
	.2 copra.rest.
	.2 copra.websocket.
}


\subsection{recorder\_full}
\subsubsection{Description}
The \texttt{recorder\_full} module contains the class \textcolor{olive}{\texttt{L2\_Update}}. \textcolor{olive}{\texttt{L2\_Update}} is responsible for:
\begin{itemize}[noitemsep]
	\item initializing instances of the order book history class, \textcolor{olive}{\texttt{history}}.
	\item interpreting the snapshot, ticker and l2update messages that come from the websocket.
	\item calling the appropriate functions and classes to carry out the orderly update to the limit order book, and order book history arrays.
\end{itemize}

\subsubsection{\textit{class} \textcolor{olive}{\texttt{L2\_Update}} }
\noindent\textbf{Description:} The class that handles messages incoming from the websocket connection. Its methods are:
\begin{itemize}[noitemsep]
	\item \textcolor{blue}{\texttt{\_\_init\_\_}}\texttt{(self, loop, channel, input\_args)}
	\item
	\textcolor{blue}{\texttt{on\_open}}\texttt{(self)}
	\item \textcolor{blue}{\texttt{on\_message}}\texttt{(self, msg)}
	\item \textcolor{blue}{\texttt{on\_close}}\texttt{(self, was\_clean, code, reason)}
\end{itemize}

\textcolor{blue}{\texttt{\_\_init\_\_}}\texttt{(self, loop, channel, input\_args)} begins by inheriting attributes \texttt{loop}, and \texttt{channel}, from \textcolor{olive}{\texttt{Client}}, and it
\begin{enumerate}[noitemsep]
	\item initializes the class \texttt{history} imported from \texttt{LOB\_funcs} and
	\item passes on the settings from the class \texttt{input\_args} and \item uses functions \textcolor{blue}{\texttt{on\_message}}\texttt{(self, msg)},
	\item \textcolor{blue}{\texttt{on\_close}}\texttt{(self, was\_clean, code, reason)} to manage incoming messages.
\end{enumerate} 
\cleardoublepage



\paragraph{\textit{function} \textcolor{blue}{\texttt{\_\_init\_\_}}\texttt{(self, loop, channel, input\_args)}}\hfill\break
\noindent \textbf{Description:} Initializes the class, using the attributes \texttt{loop, channel} from \texttt{Client} and attributes \texttt{position\_range, recording\_duration} from the class \texttt{input\_args}. It also creates an instance of the class \texttt{history}. 

\begin{tabular}{c r l }
	\textbf{parameters:}	& loop: & object\\
	&  & Comes from CoPrA\\
	&channel:& object\\
	&&comes from CoPrA\\
	&input\_args:& class\\
	&& passes on arguments for recording duration, and position range	
\end{tabular}

\begin{tabular}{l c l}
	\textbf{returns:} & None & \\
\end{tabular}

\textbf{Example: None}
%	\begin{lstlisting}[language=Python, caption=Python example]
%	 	settings = input\_args (recording_duration=5, position_range=5)
%	 	
%	 	channel1 = Channel()
%	 	channel2 = 
%	 	def hellobaby(self, yo, mama)
%	 \end{lstlisting} % DONE %
\cleardoublepage

\paragraph{\textit{function} \textcolor{blue}{\texttt{on\_open}}\texttt{(self)}}
\hfill\break
\textbf{Description:} From class \texttt{Client} uses method \texttt{on\_open}\texttt{()} to set things up. See ACoPrA \texttt{on\_open()} for more information.

\noindent \textit{restated from} \texttt{CoPrA}:\hfill\break
\fbox{\begin{minipage}{\textwidth}
		\texttt{on\_open} is called as soon as the initial WebSocket opening handshake is complete. The connection is open, but the client is not yet subscribed.
		
		If you override this method it is important that you still call it from your subclass’ \texttt{on\_open} method, since the parent method sends the initial subscription request to the WebSocket server. Somewhere in your \texttt{on\_open} method you should have \texttt{super().on\_open()}.
		
		In addition to sending the subscription request, this method also logs that the connection was opened.
\end{minipage}}

\begin{tabular}{r r l }
	\textbf{parameters:}	& None: & None\\
\end{tabular}

\begin{tabular}{r r l}
	\textbf{returns:} & None: & None\\
\end{tabular}

\textbf{Example: None} % DONE % 
\cleardoublepage

\paragraph{\textit{function} \textcolor{blue}{\texttt{on\_message}}\texttt{(self, msg)},}
\hfill\break
\textbf{Description:} 
After matching \texttt{msg['type']} to \texttt{'snapshot','ticker', 'l2update'} do one of three actions,
\begin{center}
	\begin{tabular}{ |c|c| }
		\hline
		\texttt{msg[`type'] ==} & action\\
		\hline
		\texttt{`snapshot'}&initialize the limit order book using the \texttt{initialize\_snap\_events}\\
		\hline
		\texttt{`ticker'}&  parse a market order,\\
		\hline
		\texttt{`l2update'}& parse a limit order insertion or cancellation to the limit order book.\\
		\hline
	\end{tabular}
\end{center}

\noindent \textit{restated from} \texttt{CoPrA}:\hfill\break
\fbox{\begin{minipage}{\textwidth}
		\texttt{on\_message}on is called every time a message is received. message is a dict representing the message. Its content will depend on the type of message, the channels subscribed to, etc. Please read Coinbase Pro’s WebSocket API documentation to learn about these message formats.
		
		Note that with the exception of errors, every other message triggers this method including things like subscription confirmations. Your code should be prepared to handle unexpected messages.
		
		This default method just prints the message received. If you override this method, there is no need to call the parent method from your subclass’ method.
\end{minipage}}

\begin{tabular}{r r l }
	\textbf{parameters:}	& msg: & dict\\
\end{tabular}

\begin{tabular}{r r l}
	\textbf{returns:} & None: & None\\
\end{tabular}

\textbf{Example:}
\begin{lstlisting}[language=Python]]
# Let our module be imported and our channels be assigned as follows:
import recorder_full as rec
channel1= Channel('level2', 'BTC-USD')
channel2= Channel('ticker', 'BTC-USD')


#Let the instance of the class L2_Update be ws_1 
ws_1 = rec.L2_Update(loop, channel1, settings_1)
ws_1.subscribe(channel2)

#Let the message received from the websocket be:
msg = {
	"type": "l2update",
	"product_id": "BTC-USD",
	"time": "2019-08-14T20:42:27.265Z",
	"changes": [
	[
	"buy",
	"10101.80000000",
	"0.162567"
	]
	]
}

# let the current state of the orderbook be defined as follows:
ws_1.orderbook_instance.snapshot_bid = [
	[10101.00, 5.23], [10101.50, 1.11], [10101.80, 0.5]
	]

ws_1.orderbook_instance.snapshot_signed = [
	[10101.00, -5.23], [10101.50, -1.11], [10101.80, -0.5],
	[10101.90, 0.4], [10102.00, 1.3], [10102.10, 5.00]
	]

ws_1.orderbook_instance.bid_events = []
ws_1.orderbook_instance.signed_events = []
#Suppose on loop you this message was passed to on_message(msg)
on_message)(msg)

# the new values for the the snapshots, and events would be:
>>print(ws_1.orderbook_instance.snapshot_bid)
>>[[10101.00, 5.23], [10101.50, 1.11], [10101.80, 0.162567]]

>>print(ws_1.orderbook_instance.snapshot_signed)
>>[[10101.00, -5.23], [10101.50, -1.11], [10101.80, -0.162567], [10101.90, 0.4], [10102.00, 1.3], [10102.10, 5.00]]

>>print(ws_1.orderbook_instance.bid_events)
>>[[2019-08-14T20:42:27.265Z, 10101.80, cancellation, 0.337433, 1, 10101.85, 0.10]]

>>print(ws_1.orderbook_instance.signed_events)
>>[[2019-08-14T20:42:27.265Z, 10101.80, cancellation, 0.337433, -1, 10101.85, 0.10]]
\end{lstlisting}

 % DONE % 
\cleardoublepage

\paragraph{\textit{function} \textcolor{blue}{\texttt{on\_close}}\texttt{(self, was\_clean, code, reason)}}\hfill\break
\noindent \textbf{description:} The function called when the \texttt{close()} task is executed. Begins the post processing of accumulated time-series using the functions:
\begin{itemize}
	\item \texttt{hf.}\textcolor{blue}{\texttt{convert\_array\_to\_list\_dict}}
	\item \texttt{hf.}\textcolor{blue}{\texttt{convert\_array\_to\_list\_dict\_sob}}
	\item \texttt{hf.}\textcolor{blue}{\texttt{pd\_excel\_save}}
\end{itemize}

and \texttt{gc.}\textcolor{blue}{\texttt{collect()}} to clear memory after storing \texttt{.xlsx} files of the processed data.

\noindent \textit{restated from} \texttt{CoPrA}:\hfill\break
\fbox{\begin{minipage}{\textwidth}
		\texttt{on\_close} is called whenever the connection between the client and server is closed. \texttt{was\_clean} is a boolean indicating whether or not the connection was cleanly closed. code, an integer, and reason, a string, are sent by the end that initiated closing the connection.
		
		If the client did not initiate this closure and \texttt{client.auto\_reconnect} is set to True, the client will attempt to reconnect to the server and resubscribe to the channels it was subscribed to when the connection was closed. This method also logs the closure.
		
		If your subclass overrides this method, it is important that the subclass method calls the parent method if you want to preserve the auto reconnect functionality. This can be done by including \texttt{super().on\_close(was\_clean, code, reason)} in your subclass method.	
\end{minipage}}
\cleardoublepage


\subsection{LOB\_funcs}
\subsubsection{\textbf{Description:}}Contains the class \textcolor{olive}{\texttt{history}} and the methods, \textcolor{blue}{\texttt{UpdateSnapshot\_Bid\_Seq}} and 
\textcolor{blue}{\texttt{UpdateSnapshot\_Ask\_Seq}} that outline how to update snapshots.


\subsubsection{\textit{class} \textcolor{olive}{\texttt{history()}}}\hfill \break
\textbf{Description:} Contains attributes for the time series array of the limit order book, the current snapshot of the limit order book and the time-series of the events in the limit order book. We detail the role of each attribute in the Table whateever. The class contains methods that operate of these attributes, in order to update the order book and record changes. The functions \texttt{UpdateSnapshot\_bid\_Seq} and \texttt{UpdateSnapshot\_ask\_Seq}found outside ofthese methods, dictate the sequence in which the methods contained in \texttt{history} are executed.

\paragraph{\textit{function} \textcolor{blue}{\texttt{\_\_init\_\_}}\texttt{(self)}}\hfill\break
\textbf{Description:} Initializes the lists and variables that will be operated on by other class methods. Each side, bid, ask and the combined signed has a list demarked by \texttt{\_history}, that will be populated by timestamped states of the limit order book. The lists prefixed by \texttt{snapshot} are populated by the state of limit order book demarked by the suffix \texttt{bid}, \texttt{ ask}, and \texttt{signed}. The variables \texttt{order\_type},  \texttt{position} and \texttt{event\_size} denote the type, position and size of the event based on the change in the order book as a result of an \texttt{l2\_update} message. The \texttt{token} variable determines whether the change in the order book is recorded in the events list. 
%
%\begin{itemize}
%	\item  bid\_history - list containing timeseries entries for the bid side in the form \texttt{[timestamp, snapshot\_bid]}
%	\item  ask\_history - list containing timeseries entries for the ask side in the form \texttt{[timestamp, snapshot\_ask]}
%	\item  signed\_history - list containing timeseries entries for the signed order book in the form \texttt{[timestamp, snapshot\_signed]}
%	\item snapshot\_bid - list containing current state of the order book on the bid side in the form \texttt{[[price, volm], \ldots ]}
%	\item snapshot\_bid - list containing current state of the order book on the bid side in the form \texttt{[[price, volm], \ldots ]}
%	\item snapshot\_bid - list containing current state of the order book on the bid side in the form \texttt{[[price, volm], \ldots ]}
%	\item order\_type 
%	\item token
%	\item position
%	\item event\_size 
%\end{itemize}

\begin{tabular}{r r l }
	\textbf{parameters:}	& None: & None\\
\end{tabular}

\begin{tabular}{r r l}
	\textbf{returns:} & None: & None\\
\end{tabular}

\textbf{Example: None}

\input{sections/Modules/history/history_initialize_snap_events}

\paragraph{add\_market\_order\_message}
\hfill\break
\textbf{Description:}function that parses a message object, aggregates for duplicate timestamps, and appends them to the events list. 

\paragraph{remove\_price\_level()}
\hfill \break
\textbf{Description:} Checks the level depth of an l2\_update message, removes the existing price level from the snapshot array if the level depth is 0. 
counts as a change to the limit order book therefore self.token is set to True.

\textbf{returns} an updated snapshot 

\paragraph{update\_price\_level()}
\hfill \break
\textbf{Descirption:} updates the existing price level to a new size. Sets self.order type to insertion if the new size is larger than the old size, or cancellation if the new size is smaller than the old size. sets the position to the self.position variable to the index where this event happened.

\textbf{returns} an updated snapshot

\paragraph{\textit{function} \textcolor{blue}{\texttt{update\_price\_index\_buy}}\texttt{(self, level\_depth, price\_level, pre\_level\_depth)}}\hfill\break
\noindent \textbf{Description:} If there is a new price level introduced, this function determines its location and appropriately inserts it into the snapshot bid array. It will also set the position, and token depending on where the change occurs. 

\begin{tabular}{r r l }
	\textbf{parameters:}	& level\_depth: & float64\\
	&  & level depth received from the \texttt{l2update} message\\
	& price\_level:& float64\\
	&& The price level being introduced\\
	& pre\_level\_depth:& float64\\
	&& some pre\_level\_depth that is reset to 0, idk this feature
\end{tabular}

\begin{tabular}{l c l}
	\textbf{returns:} & snapshot\_bid & list\\
	& & The modified snapshot\_bid.\\
	& bid\_range: & list \\
	&& The modified range of bids available in the snapshot.\\
	&pre\_level\_depth &float64\\
	&& again the pre level depth that idk why i still have this.\\
\end{tabular}

\textbf{Example: None}
\begin{lstlisting}[language=Python]
	x = __my_function__(1.234, True, 3)
	>>print(x)
	>>-1.23
	
	x = __my_function__(1.234, False, 3)
	>>print(x)
	>>1.23
\end{lstlisting}


\paragraph{\textit{function} \textcolor{blue}{\texttt{update\_price\_index\_sell}}\texttt{(self, level\_depth, price\_level, pre\_level\_depth)}}\hfill\break
\noindent \textbf{Description:} If there is a new price level introduced, this function determines its location and appropriately inserts it into the snapshot ask array. It will also set the position, and token depending on where the change occurs. 

\begin{tabular}{r r l }
	\textbf{parameters:}	& level\_depth: & float64\\
	&  & level depth received from the \texttt{l2update} message\\
	& price\_level:& float64\\
	&& The price level being introduced\\
	& pre\_level\_depth:& float64\\
	&& some pre\_level\_depth that is reset to 0, idk this feature
\end{tabular}

\begin{tabular}{r r l}
	\textbf{returns:} & snapshot\_ask: & list\\
	& & The modified snapshot\_ask.\\
	& bid\_range: & list \\
	&& The modified range of bids available in the snapshot.\\
	&pre\_level\_depth &float64\\
	&& again the pre level depth that idk why i still have this.\\
\end{tabular}

\begin{tabular}{r r l}
	\textbf{latent changes:} & token: & boolean\\
	& & bool that informs whether a significant change has occurred in the order book. \\
	& order\_type:& string\\
	&& type of order in \texttt{[`market', `insertion', `cancelation']}\\
	
\end{tabular}


\textbf{Example:}
\begin{lstlisting}[language=Python]
	# let our snapshot_bid be:
	snapshot_ask =  [[10101.90, 0.013400],[10102.00, 0.52100], [10102.11, 0.89041]]
	
	# we receive a 'l2update' message from our event loop
	
	msg = {
		"type": "l2update",
		"product_id": "BTC-USD",
		"time": "2019-08-14T20:42:27.265Z",
		"changes": [
		[
		"sell",
		"10101.95000000",
		"0.500000"
		]
		]
	}
	
	# on_message() conventiently parses the message to create relevant local variables
	# time, changes, side, price_level, level_depth, pre_level_depth, hist.token
	>>print(time, side,price_level, level_depth, prelevel_depth, hist.token)
	>>2019-08-14T20:42:27.265Z "buy" 10101.90000000 0.500000 0 False
	
	# in on_message(), the variable price_match_index is assigned by finding the index
	# with the matching price, in the snapshot. In this case:
	price_match_index = [] 
	
	snapshot_ask, pre_level_depth = update_price_index_sell(level_depth, price_level, pre_level_depth)
	
	>>print(snapshot_bid, hist.token)
	> [[10101.90, 0.013400],[10101.95, 0.500000],[10102.00, 0.52100], [10102.11, 0.89041]] True
\end{lstlisting}

\paragraph{\textit{function} \textcolor{blue}{\texttt{\_\_my\_function\_\_}}\texttt{(self)}}\hfill\break
\noindent \textbf{Description:} Updates the variable \texttt{bid\_range}. I'm not sure why I need to do this but it does keep update snapshot bid/ask from making mistakes when doing bisect/insert. 


\begin{tabular}{r r l }
	\textbf{parameters:}	& None: & None\\
	&  & None\\
\end{tabular}

\begin{tabular}{l c l}
	\textbf{returns:} & snapshot\_bid: & list\\
	& & The snapshot on the bid side. \\
	& bid\_range: & list\\
	&& ordered range of prices for the bid side \\
\end{tabular}

\begin{tabular}{l c l}
	\textbf{latent changes:} & None & None\\
	& &  None
\end{tabular}

\textbf{Example: None}
\begin{lstlisting}[language=Python]
	x = __my_function__(1.234, True, 3)
	>>print(x)
	>>-1.23
	
	x = __my_function__(1.234, False, 3)
	>>print(x)
	>>1.23
\end{lstlisting}

\paragraph{\textit{function} \textcolor{blue}{\texttt{update\_snapshot\_ask}}\texttt{(self)}}\hfill\break
\noindent \textbf{Description:} Updates the variable \texttt{ask\_range}. I'm not sure why I need to do this but it does keep update snapshot bid/ask from making mistakes when doing bisect/insert. 


\begin{tabular}{r r l }
	\textbf{parameters:}	& None: & None\\
	&  & None\\
\end{tabular}

\begin{tabular}{l c l}
	\textbf{returns:} & snapshot\_ask: & list\\
	& & The snapshot on the ask side. \\
	& ask\_range: & list\\
	&& ordered range of prices for the ask side \\
\end{tabular}

\begin{tabular}{l c l}
	\textbf{latent changes:} & None & None\\
	& &  None
\end{tabular}

\textbf{Example: None}
\begin{lstlisting}[language=Python]
	?????????
\end{lstlisting}

\input{sections/Modules/history/history_trim_coordinator}

\paragraph{\textit{function} \textcolor{blue}{\texttt{append\_snapshot\_bid}}\texttt{(self, time, price\_level)}}\hfill\break
\noindent \textbf{Description:} Appends the current form of the snapshot to \texttt{bid\_history} and computes the event and appends the event record to \texttt{bid\_events}. 

\begin{tabular}{r r l }
	\textbf{parameters:}	& : & float64\\
	&  & some input that will be passed to the function\\
	& param\_2:& boolean\\
	&& a conditional that will be checked by the function\\
	& param\_3:& int\\
	&& some int that is needed for the function	
\end{tabular}

\begin{tabular}{l c l}
	\textbf{returns:} & trans\_number & float64\\
	& & The number with the correct significant digits and sign. 
\end{tabular}

\begin{tabular}{l c l}
	\textbf{latent changes:} & trans\_number & float64\\
	& & The number with the correct significant digits and sign. 
\end{tabular}

\textbf{Example: None}
\begin{lstlisting}[language=Python]
	x = __my_function__(1.234, True, 3)
	>>print(x)
	>>-1.23
	
	x = __my_function__(1.234, False, 3)
	>>print(x)
	>>1.23
\end{lstlisting}

\input{sections/Modules/history/history_append_snapshot_ask}

\paragraph{\textit{function} \textcolor{blue}{\texttt{append\_signed\_book}}\texttt{(self, time, price\_level)}}\hfill\break
\noindent \textbf{Description:} Appends the current form of the snapshot to \texttt{ask\_history} and computes the event and appends the event record to \texttt{ask\_events}. 

\begin{tabular}{r r l }
	\textbf{parameters:}	& time: & datetime object\\
	&  & timestamp that when the message was received\\
	& price\_level:& float64\\
	&& price level where the change occurred\\
\end{tabular}

\begin{tabular}{l c l}
	\textbf{returns:} & snapshot\_ask & list\\
	& & The snapshot on the ask side.\\
	& ask\_range & list\\
	&& the range of prices in \texttt{snapshot\_ask}.
\end{tabular}

\begin{tabular}{l c l}
	\textbf{latent changes:} & None: & None\\
\end{tabular}

\textbf{Example: None}
\begin{lstlisting}[language=Python]
	x = __my_function__(1.234, True, 3)
	>>print(x)
	>>-1.23
	
	x = __my_function__(1.234, False, 3)
	>>print(x)
	>>1.23
\end{lstlisting}

\paragraph{\textit{function} \textcolor{blue}{\texttt{chk\_mkt\_can\_overlap}}\texttt{(self, events, order\_type)}}\hfill\break
\noindent \textbf{Description:} passes the order\_type either market order or canceled order. If it was a market order checks for a recent canceled order of the same size and delectes the canceled order. likewise, if the ordertype passedwas a canceled order it looks for a recent market order and deletes the canceled order. 

The reason this is done this way is because the l2update channel only updates queue sizes while the ticker channel only announces market order arrivals. 
 

\begin{tabular}{r r l }
	\textbf{parameters:}	& events: & list\\
	&  & current event list to be modified\\
	& order\_type:& string\\
	&& type of order observed \texttt['market','cancellation']\\
\end{tabular}

\begin{tabular}{l c l}
	\textbf{returns:} & events & list\\
	& & modified event list\\
\end{tabular}

\begin{tabular}{l c l}
	\textbf{latent changes:} & None: & None\\
\end{tabular}

\textbf{Example: None}
\begin{lstlisting}[language=Python]
	x = __my_function__(1.234, True, 3)
	>>print(x)
	>>-1.23
	
	x = __my_function__(1.234, False, 3)
	>>print(x)
	>>1.23
\end{lstlisting}

\noindent\textbf{Accessors:} The folowing functions, prefaced with \texttt{last\_} are the currently named accessors. They cover the last order of a named type, the last state of the order book, and the market depth for a the last given state of the order book. If I come up with other relevant accessors or refine how each function works their changes will likely be noted in the versions, or at least a git entry. 

\paragraph{\textit{function} \textcolor{blue}{\texttt{last\_inserted\_order}}\texttt{(self, side=``signed'')}}\hfill\break
\noindent \textbf{Description:} Returns the latest limit order insertion from the bid, ask, or signed events. Defaults to signed. 

\begin{tabular}{r r l }
	\textbf{parameters:}	& side: & string\\
	&  & market side to query \\
\end{tabular}

\begin{tabular}{l c l}
	\textbf{returns:} & out(please rename) & list\\
	& & The last inserted event. 
\end{tabular}

\begin{tabular}{l c l}
	\textbf{latent changes:} & None: & None\\
\end{tabular}

\textbf{Example: None}
\begin{lstlisting}[language=Python]
	x = __my_function__(1.234, True, 3)
	>>print(x)
	>>-1.23
	
	x = __my_function__(1.234, False, 3)
	>>print(x)
	>>1.23
\end{lstlisting}


\subsection{history\_funcs.py}\hfill \break
\noindent \textbf{Description:}Helper functions that operate in recorder\_full.py and LOBfuncs.py. It contains simple methods that are agnostic to side (e.g., bid/ask/signed) for the most part.  

\paragraph{\textit{function} \textcolor{blue}{\texttt{check\_order}}\texttt{(snapshot, side)}}\hfill\break
\noindent \textbf{Description:} Checks that the snapshot is sorted correctly. Used in debugging when snapshots do not look well ordered. Only checks bid and ask side, because signed snapshot is a concatenated form of a bid and ask snapshot. \texttt{pass} placeholder can be changed to a print statement. 

\begin{tabular}{r r l }
	\textbf{parameters:}	& snapshot: & list\\
	&  & snpshot to check\\
	& side:& string \\
	&& side \texttt{bid, ask}	
\end{tabular}

\begin{tabular}{l c l}
	\textbf{returns:} & snapshot & list\\
	& & 
\end{tabular}

\begin{tabular}{l c l}
	\textbf{latent changes:} & None: & None\\
\end{tabular}

\textbf{Example: None}
\begin{lstlisting}[language=Python]
	# let our snapshot_bid be:
	snapshot_bid =  [[10101.80, 0.013400],[10101.20, 0.44100], [10101.10, 0.450541]]
	
	#I will say that if the snapshot is not ordered I can ask the function to return a boolean, True if ordered, else False
	>>x = check order(snapshot_bid, "bid")
	>>print(x)
\end{lstlisting}
\cleardoublepage

\input{sections/Modules/history_funcs/convert_array_to_list_dict}
\cleardoublepage

\paragraph{\textit{function} \textcolor{blue}{\texttt{conver\_array\_to\_list\_dict\_sob}}\texttt{(history, events, pos\_limt=5)}}\hfill\break
\noindent \textbf{Description:} converts the time-series of order book states into a list of dictionaries for the signed order book. This has some nuance because the signed book format has negative volumes, and the ordinal scale is prefaced by a negative sign. Works similarly to \textcolor{blue}{\texttt{conver\_array\_to\_list\_dict}} but uses the mid-price from \texttt{events} to find where to separate the bid and ask sides. 

\begin{tabular}{r r l }
	\textbf{parameters:}	& history: & list\\
	&  & time series of order book states\\
	& events:& list \\
	&&The events list where the function will extract the mid price from.\\
	& pos\_limit:& int, default := 5\\
	&& Ordinal distance from the best bid/ask that is worth saving.	
\end{tabular}

\begin{tabular}{l c l}
	\textbf{returns:} & volm\_list: & list\\
	& & list of dictionaries for volume sizes and their respective position.\\
	& price\_list: & list\\
	&& list of dictionaries for prices and their respective position.
\end{tabular}

\begin{tabular}{l c l}
	\textbf{latent changes:} & None: & None\\
\end{tabular}

\textbf{Example: None}
\begin{lstlisting}[language=Python]
	# let our snapshot_bid be:
	bid_history = [
	[2019-08-14T20:42:27.265Z,[[10101.80, 0.013400],[10101.20, 0.44100], [10101.10, 0.450541], [10100.55, 5.24501], [10099.00, 10.24511], [10090.11, 24.21395]]],
	[2019-08-14T20:42:27.500Z,[[10101.20, 0.44100], [10101.10, 0.450541], [10100.55, 5.24501], [10099.00, 10.24511], [10090.11, 24.21395],[10090.05]] ],
	[2019-08-14T20:42:27.963Z,[[10101.20, 0.55200], [10101.10, 0.450541], [10100.55, 5.24501], [10099.00, 10.24511], [10090.11, 24.21395]]]
	]
	
	ordinal_volumes, ordinal_prices = convert_array_list_to_dict(snapshot)
	
	>>print(ordinal_volumes)
	>>[{"time":2019-08-14T20:42:27.265Z, "1":0.013400, "2":0.44100, "3":0.450541, "4":5.24501, "5":10.24511},
	{"time":2019-08-14T20:42:27.500Z, "1":0.44100, "2":0.450541, "3":5.24501, "4":10.24511, "5":24.21395},
	{"time":2019-08-14T20:42:27.963Z, "1":0.52200, "2":0.450541, "3":5.24501, "4":10.24511, "5":2421395}]
	
	>>print(ordinal_prices)
	>>[{"time":2019-08-14T20:42:27.265Z, "1":10101.80, "2":10101.20, "3":10101.10, "4":10100.55, "5":10099.00},
	{"time":2019-08-14T20:42:27.500Z, "1":10101.20, "2":10101.10, "3":10100.55, "4":10099.00, "5":10090.11},
	{"time":2019-08-14T20:42:27.963Z, "1":10101.20, "2":10101.10, "3":10100.55, "4":10099.00, "5":10090.11}]
\end{lstlisting}
\cleardoublepage

\paragraph{\textit{function} \textcolor{blue}{\texttt{pd\_excel\_save}}\texttt{(title, hist\_obj\_dict)}}\hfill\break
\noindent \textbf{Description:} Convert the list of dictionary objects into a pandas dataframe object. Excises the first entry, because it is usually empty, saves the dataframe object as a .xlsx file with the title given as an argument. 

\begin{tabular}{r r l }
	\textbf{parameters:}	& title: & string\\
	&  & Title of the output .xlsx file\\
	& hist\_obj\_dict:& list\\
	&& The list of dictionaries to be converted to a pandas dataframe object.\\
\end{tabular}

\begin{tabular}{l c l}
	\textbf{returns:} & None: & None\\
\end{tabular}

\begin{tabular}{l c l}
	\textbf{latent changes:} & None: & None\\
\end{tabular}

\textbf{Example: None}
\begin{lstlisting}[language=Python]
ordinal_volumes = [{"time":2019-08-14T20:42:27.265Z, "1":0.013400, "2":0.44100, "3":0.450541, "4":5.24501, "5":10.24511},
{"time":2019-08-14T20:42:27.500Z, "1":0.44100, "2":0.450541, "3":5.24501, "4":10.24511, "5":24.21395},
{"time":2019-08-14T20:42:27.963Z, "1":0.52200, "2":0.450541, "3":5.24501, "4":10.24511, "5":2421395}]

volm_title =  "L2_orderbook_volm_bid"+str(bid_events[-1]['time'])+".xlsx"

>>pd_excel_save(volm_title, ordinal_volumes)
\end{lstlisting}

\noindent Below a simple table generated from L2\_orderbook\_volm\_bid2019-08-14T20:42:27.963Z.xlsx is shown.

\begin{center}
	\begin{tabular}{|c|c|c|c|c|c|}
		\hline
		time & 1 & 2 & 3 & 4 & 5\\
		\hline
		2019-08-14T20:42:27.265 & 0.013400 & 0.44100 & 0.450541 & 5.24501 & 10.24511 \\
		\hline
		2019-08-14T20:42:27.500 & 0.44100 & 0.450541 & 5.24501 & 10.24511 & 24.21395 \\
		\hline
		2019-08-14T20:42:27.963 & 0.52200 & 0.450541 & 5.24501 & 10.24511 & 24.21395 \\
		\hline
	\end{tabular}
\end{center} 
\cleardoublepage

\paragraph{\textit{function} \textcolor{blue}{\texttt{set\_sign}}\texttt{(event\_size, side, order\_type)}}\hfill\break
\noindent \textbf{Description:} Sets the multiplicative sign,either 1 or -1 depending on the order type and event side. 

\begin{tabular}{r r l }
	\textbf{parameters:}	& event\_size: & float64\\
	&  & Size of  the event, artifact from when I returned the signed event\\
	& side:& string\\
	&& Event side either "bid" or "ask"\\
	& order\_type:& str\\
	&& the type of order either \texttt{``insertion''}, \texttt{``cancellation''} or \texttt{``market''}.
\end{tabular}

\begin{tabular}{l c l}
	\textbf{returns:} & sign & int\\
	& & The correct sign given the event's order type and side.
\end{tabular}

\begin{tabular}{l c l}
	\textbf{latent changes:} & None: & None\\
\end{tabular}

\textbf{Example: None}
\begin{lstlisting}[language=Python]
	x = __my_function__(1.234, True, 3)
	>>print(x)
	>>-1.23
	
	x = __my_function__(1.234, False, 3)
	>>print(x)
	>>1.23
\end{lstlisting}
\cleardoublepage

\paragraph{\textit{function} \textcolor{blue}{\texttt{set\_signed\_positiion}}\texttt{(position, side)}}\hfill\break
\noindent \textbf{Description:} Simple conditional that sets the position for the bid side.... I'm tired...

\begin{tabular}{r r l }
	\textbf{parameters:}	& param\_1: & float64\\
	&  & some input that will be passed to the function\\
	& param\_2:& boolean\\
	&& a conditional that will be checked by the function\\
	& param\_3:& int\\
	&& some int that is needed for the function	
\end{tabular}

\begin{tabular}{l c l}
	\textbf{returns:} & trans\_number & float64\\
	& & The number with the correct significant digits and sign. 
\end{tabular}

\begin{tabular}{l c l}
	\textbf{latent changes:} & trans\_number & float64\\
	& & The number with the correct significant digits and sign. 
\end{tabular}

\textbf{Example: None}
\begin{lstlisting}[language=Python]
	x = __my_function__(1.234, True, 3)
	>>print(x)
	>>-1.23
	
	x = __my_function__(1.234, False, 3)
	>>print(x)
	>>1.23
\end{lstlisting}
\cleardoublepage

\paragraph{\textit{function} \textcolor{blue}{\texttt{get\_min\_dec}}\texttt{(min\_currency\_denom, min\_asset\_value)}}\hfill\break
\noindent \textbf{Description:} determines the decimal place for the smallest amount of base currency needed for the minimal denomination of the float currency. 

\noindent Example:  Suppose you have an exchange rate of $10010.99 USD/BTC$, (i.e., you need $10010.99 USD$ to buy $1 BTC$) then the amount of BTC in 1 cent or 0.01 USD which is the smallest denomination of the USD is $0.000000999 BTC$. We see that this is $10^{-7}$,  \textcolor{blue}{\texttt{get\_min\_dec}} would return $7$, to pass onto \texttt{numpy.around}.


\begin{tabular}{r r l }
	\textbf{parameters:}	& min\_currency\_denom : & float64\\
	&  & the minimum currency denomination (e.g., 0.01 USD)\\
	& min\_asset\_value:& float646\\
	&& the smallest current valuation of the asset.\\
\end{tabular}

\begin{tabular}{l c l}
	\textbf{returns:} & min\_dec\_out & int\\
	& & The number of decimal places out for the smallest movement of the asset for a minimal denomination ..sdfkljfsdlka rewwrite.  
\end{tabular}

\begin{tabular}{l c l}
	\textbf{latent changes:} & None: & None\\
\end{tabular}

\textbf{Example: None}
\begin{lstlisting}[language=Python]
??????


\end{lstlisting}
\cleardoublepage
