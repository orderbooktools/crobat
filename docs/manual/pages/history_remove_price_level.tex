\paragraph{\textit{function} \textcolor{blue}{\texttt{remove\_price\_level}}\texttt{(snap\_array, level\_depth, match\_index)}}
\hfill \break
\textbf{Description:} Checks the level depth of an \texttt{l2\_update} message, removes the existing price level from the snapshot array if the level depth is 0. Counts as a change to the limit order book therefore self.token is set to True. 

\begin{tabular}{r r l }
	\textbf{parameters:}	& snap\_array & array\\
	& level\_depth & float64\\
	& match\_index& int\\ 
\end{tabular}

\begin{tabular}{r r l}
	\textbf{returns:} & snap\_array: & array\\
\end{tabular}

\textbf{Example:}
\begin{lstlisting}[language=Python]
	# let our snapshot_bid be:
	snapshot_bid =  [[10101.80, 0.013400],[10101.20, 0.44100], [10101.10, 0.450541]]
	
	# we receive a 'l2update' message from our event loop
	
	msg = {
	"type": "l2update",
	"product_id": "BTC-USD",
	"time": "2019-08-14T20:42:27.265Z",
	"changes": [
	[
	"buy",
	"10101.80000000",
	"0.0"
	]
	]
	}
	
	# on_message() conventiently parses the message to create relevant local variables
	# time, changes, side, price_level, level_depth, pre_level_depth, hist.token
	>>print(time, side,price_level, level_depth, prelevel_depth, hist.token)
	>>2019-08-14T20:42:27.265Z "buy" 10101.80000000 0.0 0 False
		
	# in on_message(), the variable price_match_index is assigned by finding the index
	# with the matching price, in the snapshot. In this case:
	price_match_index = [0] 
	
	snapshot_bid = remove_price_level(snapshot_bid, level_depth, price_match_index)
	
	>>print(snapshot_bid, hist.token)
	>>[[10101.20, 0.44100], [10101.10, 0.450541]] True
\end{lstlisting}

